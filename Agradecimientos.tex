%----------------------------------------------------------------------------------------
%	AGRADECIMIENTOS
%----------------------------------------------------------------------------------------

\chapter*{Agradecimientos}
%\markboth{AGRADECIMIENTOS23}{AGRADECIMIENTOS} % encabezado 

Cuando se consiguen logros como este uno quiere agradecerle a todo el mundo, la lista es interminable e insólita, así que me voy a olvidar de muchos.
No creo que lo noten.

Primero que nada, a mi familia. Desde Lorena que me banca las noches sin dormir, que me levante tratando de no hacer ningún ruido a la madrugada por que ``tengo algo dando vueltas en la cabeza que no me deja dormir'', Giuli y Luca, que me obligan a hacer ejercicios de concentración cuando trato de escribir algún código con alguno a upa. Mi vieja, Sonia, de la que heredé esta forma de pensar tan lateral y mi viejo, Eduardo, que me metió de prepo en la cabeza este bicho de querer estudiar, curiosear, saber por qué... males que comparto con mis hermanos Anibal y Adriana que terminaron siendo una compañía en esto de ser un poco raro.

Por supuesto, a mis compañeros de trabajo, principalmente a mis directoras: Luciana, que me obliga a moverme cuando hay que hacer algo que no me gusta, dejándome esa sensación de que si no termino se termina el mundo e Hilda con la que lamentablemente pude compartir pocos años, persona a la que admiro y a la que traté de copiar en esa claridad para plantearse objetivos. Uno termina laburando mucho y feliz cuando estuvo un rato con Hilda.
A mis compañeros de Laboratorio, Daniel, Karina, Miguel, Gustavo y Juan, con quienes se puede charlar de cualquier tema con niveles intelectuales insospechados. Hacen que la estadía sea amena.

No quisiera olvidarme que sin este sistema Universitario libre y gratuito no hubiera podido conseguir ninguno de mis logros profesionales. Un sistema en donde cualquier persona puede tener acceso a educación de primera calidad tiene que ser defendido a capa y espada, si es que alguna retribución pidiere. No puedo dejar de agradecer a los que lo pensaron, a los que lo sostienen y a los que lo defienden.