\section{Analysis of the digital implementation of a chaotic deterministic-stochastic attractor (EAMTA 2012)}

Otro que no tengo el latex, se lo tengo que pedir a Luciana

In this work the implementation, of chaos-based
pseudo random number generators (PRNG), onto a Field Programmable Gate Array (FPGA), is analyzed. Any digital implementation requires the choice of an algorithm to discretize
time and a representation standard to represent real numbers.
Each choice modifies the stochasticity degree of the system and
also defines a different amount of resources on the FPGA. The
main contribution of this paper is to propose an optimum design
methodology for applications in which the chaotic system is going
to replace a stochastic system. This is the case with PRNG. In
stochastic systems the randomness degree must be measured. In
this paper we use the global indicator proposed by Marsaglia
in his widely used DIEHARD tests-suite. Results are exemplified
for the Lorenz chaotic oscillator but the same methodology may
be used with other low dimensional chaotic systems.