En los últimos años se ha establecido que existen sistemas deterministas que rompen con el preconcepto de que los sistemas físicos pueden clasificarse en dos conjuntos disjuntos: sistemas deterministas y sistemas estocásticos.
El concepto antiguo era que un sistema determinista es aquél para el cual conocemos el modelo y por lo tanto es posible predecir con exactitud la evolución de sus variables de estado.
Se utilizan en su descripción ecuaciones diferenciales o de recurrencia.
Por otra parte un sistema estocástico es aquél para el cual el modelo no se conoce o se lo supone sumamente complejo como para ser obtenido, de modo que se adopta la estrategia de estudiar sus variables de estado en forma estadística.
Se utilizan entonces en la descripción ecuaciones diferenciales o de recurrencia estocásticas.

El caos determinista demostró que complejidad en la evolución temporal no es sinónimo de complejidad en el modelo, cuando hay alinealidad: modelos deterministas muy simples originan señales de aspecto estocástico.
La sensibilidad a las condiciones iniciales hace que en estos sistemas la predictibilidad sea a corto plazo (luego de un tiempo finito es imposible predecir la evolución) lo que ubica a estos sistemas en una posición intermedia entre determinista y estocástico \cite{Liao2013a}.

Como consecuencia se desarrollaron en los últimos años un número creciente de aplicaciones de los sistemas caóticos, empleándolos principalmente como generadores de ruido controlado \cite{DeMicco2007C}, generadores de números pseudoaleatorios \cite{DeMicco2007A}, portadoras de señales \cite{DeMicco2007B}, sistemas de encriptado \cite{Machado2004, Smaoui2009}, etc.

Hoy en día, los sistemas dinámicos son un objeto de estudio interdisciplinario, aunque originalmente fue una rama de la física.
Todo comenzó a mediados del 1600, cuando Newton inventó las ecuaciones diferenciales, descubriendo sus leyes del movimiento de gravitación universal, y las combinó con las leyes de Kepler sobre el movimiento planetario.
Específicamente, Newton resolvió el problema de los dos cuerpos (por ejemplo, el sistema tierra-sol).

Subsecuentes generaciones de matemáticos y físicos intentaron extender los métodos analíticos de Newton al problema de los tres cuerpos (por ejemplo, luna-tierra-sol), pero curiosamente para resolver este problema se necesitó mucho más esfuerzo.
Luego de décadas, se dieron cuenta de que el problema de los tres cuerpos era esencialmente imposible de resolver, en el sentido de obtener las fórmulas explícitas.

La ruptura vino con el trabajo de Poincaré a finales del 1800.
Él introdujo un nuevo punto de vista que enfatizaba las cuestiones cualitativas más que las cuantitativas (por ejemplo, ¿es estable el sistema luna-tierra-sol?).
Poincaré desarrolló una poderosa aproximación geométrica que hoy es usada para estudiar sistemas dinámicos y también fue el primero en vislumbrar la posibilidad del caos, en el cual un sistema determinístico exhibe un comportamiento aperiódico que depende sensiblemente de las condiciones iniciales, haciendo así imposible la predicción a largo plazo.

Pero el caos se mantuvo en segundo plano hasta la segunda mitad del 1900, en donde los osciladores no lineales jugaron un rol vital en el desarrollo de tecnologías de radio, radar, lazos de enganche de fase y láser.
Por el lado matemático, los osciladores no lineales también estimularon la invención de nuevas técnicas matemáticas.
Los métodos geométricos de Poincaré se fueron extendiendo para producir un conocimiento mucho más profundo de la mecánica clásica.

La invención de la computadora por el 1950 fue una línea divisoria en la historia de los sistemas dinámicos.
La computadora nos permite experimentar con ecuaciones en una forma que antes era imposible, y así explorar la dinámica los sistemas no lineales de una forma mucho más directa.
Estos experimentos llevaron a Lorenz a descubrir en 1963 el movimiento caótico de un atractor extraño, mientras estudiaba un modelo simplificado de la circulación de convección para comprender mejor la notoria impredictibilidad del clima.
Lorenz encontró que la solución a sus ecuaciones nunca caían al equilibrio o a un estado periódico.
Además, si comenzaba sus simulaciones de dos condiciones iniciales ligeramente diferentes, los comportamientos resultantes pronto serían
totalmente diferentes.
Como consecuencia de ello, el sistema es inherentemente impredecible, pequeños errores en las mediciones del estado actual de la atmósfera (o cualquier sistema caótico) sería amplificado rápidamente.
Pero Lorenz también mostró que había estructura en el caos, cuando las soluciones fueron dibujadas en tres dimensiones, las soluciones a sus ecuaciones cayeron sobre un conjunto de puntos en forma de mariposa.
Él sostuvo que este sistema tenía que ser “un infinito complejo de superficies”, lo que hoy podríamos considerar como un ejemplo de fractal.

El trabajo de Lorenz tuvo un pequeño impacto hasta 1970, los años del boom del caos.
Se desarrollaron teorías completamente nuevas basadas en consideraciones sobre atractores caóticos, como turbulencia de fluidos y biología de las poblaciones y se encontraron comportamientos caóticos en reacciones químicas \cite{Kapral1995}, osciladores mecánicos \cite{Awrejcewicz2003}, semiconductores \cite{Scholl2001} y oscilaciones biológicas como el ritmo cardíaco y circadiano \cite{Strogatz2018}.
Hoy, la teoría del caos es una herramienta más para el estudio de sistemas dinámicos y los sistemas caóticos son utilizados en una gran cantidad de dispositivos.

En este Capítulo se revisan los conceptos de espacio de fases, pasando por las soluciones típicas de sistemas de ecuaciones diferenciales, para luego poder entrar a la descripción de sistemas caóticos. Primero se abordan los sistemas caóticos continuos con derivada continua y presentamos tres ejemplos clásicos en la literatura. Luego se hace una reseña a los mapas caóticos, en donde presentamos los mapas cuadráticos bidimensionales, los cuales usaremos en algunas secciones subsiguientes.
