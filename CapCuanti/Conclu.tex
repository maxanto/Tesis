\section{Conclusiones}

En este Capítulo se presentaron las principales herramientas utilizadas para detectar caos y cuantificar la calidad estadística de los generadores de números aleatorios.
Junto con la introducción teórica, se mostraron algunos avances en la implementación de dichas herramientas.

El algoritmo evolutivo desarrollado detecta con precisión el máximo MLE del sistema en cada región en el espacio de parámetros del conocido oscilador Logístico.
La búsqueda exhaustiva del MLE barriendo todos los valores de parámetros se vuelve muy complicada cuando aumenta el número de parámetros.
Esta es la razón por la cual se empleó un algoritmo genético en este trabajo.
Este algoritmo heurístico permite encontrar las áreas de interés, $ MLE> 0 $, de una manera más rápida y simple.
En la implementación de hardware del cálculo MLE, hemos explotado la naturaleza paralela de las ecuaciones de cálculo del MLE con el objetivo de optimizar el diseño de arquitectura propuesto, permitiendo su implementación concurrente basada en tecnología FPGA.

Se desarrolló e implementó un sistema que permite medir con buena precisión las entropías causal y no-causal de señales analógicas provenientes del exterior de la FPGA y también internas generadas por código.
Se logró medir señales y realizar cálculos complejos con un microcontrolador modesto como el 8051 instanciado en la FPGA AFS1500 de ACTEL.
Este prototipo cumple con las especificaciones de precisión y cantidad de recursos requeridos establecidas en el diseño.
El código de este sistema ocupa el $15,4\%$ del total de la memoria flash del microprocesador instanciado.
En cuanto a los recursos disponibles en la FPGA se utilizaron $7~349$ celdas lógicas, quedando casi el $80\%$ de los recursos de \textit{hardware} disponibles para implementar los sistemas bajo prueba en forma concurrente.

También se exploraron las fuentes de error del medidor de entropías implementado en FPGA.
Para este primer análisis se evaluó qué sucede al aplicar un filtro abrupto, es por esto que elegimos para comparar un filtro elíptico y uno ideal. 
Las respuestas del filtro elíptico y del ideal fueron muy similares en el rango de frecuencias en los que el elíptico tiene un buen comportamiento, sin embargo cuando la frecuencia de corte del elíptico se acerca a los extremos (es decir cuando $f_c \to 0$ o $f_c \to 1$) la salida del filtro diverge.
El problema se debe a que el método numérico utilizado para calcular la salida del filtro diverge por la precisión finita utilizada.
Como todo el análisis se realiza en el dominio digital, no necesitamos volver a la frecuencia continua, por lo que nos quedamos con los resultados del filtro ideal para hacer las pruebas, sin tener que preocuparnos por el ripple que aparece en las bandas de paso y rechazo cuando pasamos al mundo analógico.
Cuando se compararon las respuestas de los cuantificadores con y sin ruido, se vio que las señales limpias tienen mesetas, es decir que se mantienen constantes hasta que el filtrado elimina la siguiente componente espectral.
Sin embargo, cuando están contaminadas con ruido los cuantificadores cambian para parecerse más a los resultados que arroja el ruido blanco gaussiano sin ninguna señal determinística.
En todos los casos se vio que estos cuantificadores son muy sensibles a la presencia de ruido, lo que nos permitió vincular a este hecho los errores en la medición.
También se vio que los valores cambian a medida que se filtra la señal sin contaminar, lo que agrega una segunda fuente de error dada por el ancho de banda finito del sistema.