\section{Introducción}

En los últimos años, los sistemas digitales se convirtieron en el estándar en todas las ciencias experimentales.
Mediante el uso de nuevos dispositivos electrónicos programables como DSP y electrónica reconfigurable como FPGA o ASIC, los experimentadores pueden diseñar y modificar sus propios generadores de señales, sistemas de medición, modelos de simulación, etc.

En estos nuevos dispositivos, el punto flotante y el punto fijo son las aritméticas más comunes.
El punto flotante es la solución más precisa cuando no se conoce el rango de valores que se implementará, pero no siempre se recomienda cuando se requieren velocidad, baja potencia y/o área de circuito pequeño, una solución de punto fijo es mejor en estos casos.

El efecto de la discretización numérica sobre un mapa caótico fue abordado recientemente en \cite{DeMicco2017}, en donde se explora la degradación estadística del espacio de fases para una familia de mapas cuadráticos 2D.
Estos mapas presentan una dinámica multiatractor que los hace muy atractivos como generadores de números aleatorios en campos como criptografía, codificación, etc. En \cite{Tlelo-Cuautle2016} y \cite{DelaFraga2017}, los autores propusieron usar el valor de la entropía para elegir el número de bits en la parte fraccionaria, cuando se implementan mapas en aritmética entera.

En \cite{Grebogi1988}, Grebogi y colaboradores estudiaron este tema y vieron que el período $T$ escala con el redondeo $\epsilon$ como $T\sim\epsilon^{-d/2}$ donde $d$ es la dimensión de correlación del atractor caótico.
Conseguir un período grande $T$ es una propiedad importante de los mapas caóticos, en \cite{Nagaraj2008} Nagaraj \textit{et al.} estudiaron el efecto de cambiar las longitudes de período promedio de los mapas caóticos en precisión finita.
Vieron que el período $T$ del mapa compuesto obtenido al conmutar entre dos mapas caóticos es mayor que el período de cada mapa.
Liu \textit{et al.} \cite{Liu2006} estudiaron diferentes reglas de conmutación aplicadas a sistemas lineales para generar caos.
El problema de la conmutación también se trató en \cite{Gluskin2008}, el autor consideró algunos aspectos matemáticos, físicos y de ingeniería relacionados con sistemas singulares, principalmente de conmutación.
Los sistemas conmutados surgen naturalmente en la electrónica de potencia y en muchas otras áreas de la electrónica digital.

La estocasticidad y la mezcla también son relevantes para caracterizar un comportamiento caótico.
Para investigar estas propiedades, se estudiaron varios cuantificadores \cite{DeMicco2009}.
La entropía y la complejidad de la teoría de la información se aplicaron para dar una medida de la entropía causal y no causal y la complejidad causal.