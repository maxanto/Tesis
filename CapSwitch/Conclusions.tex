\section{Conclusions}\label{sec:conclusions}

Exploramos la degradación estadística debido al error inherente de los sistemas en base 2 para de mapas caóticos simples, conmutados y con skipping.
Evaluamos las distribuciones de mezcla y amplitud desde un punto de vista estadístico.

Este trabajo complementa los resultados anteriores dados en \cite{Nagaraj2008}, donde se investigaron las duraciones de los períodos.
En ese sentido, nuestros resultados fueron compatibles.
Podemos ver que la conmutación entre dos mapas aumenta la dependencia del período en función de la precisión, esto se debe a que la longitud de correlación también se incrementa.
Sin embargo, el procedimiento estándar de skipping reduce la duración del período en casi la mitad.

Todas las estadísticas de los mapas representados en punto fijo producen una evolución no monótona hacia los resultados de coma flotante.
Este resultado es relevante porque muestra que no siempre se recomienda aumentar la precisión.

Es especialmente interesante observar que algunos sistemas (TENT) con muy buenas propiedades estadísticas en el mundo de los números reales, se vuelven ``patológicos'' cuando se usan representaciones numéricas binarias.
Como regla general, si un mapa se genera solo por operaciones de shifteo (esto depende de la base de la unidad lógica aritmética y del mapa en sí), todas las condiciones iniciales convergerán a un punto fijo con un transitorio no mayor que la longitud de la mantisa que está siendo utilizada.

Al comparar los cuantificadores BP y BPW, pudimos detectar caídas a puntos fijos y pudimos estimar la longitud relativa de sus transitorios.
Esto puede verse en todas las implementaciones de TENT, en una condición inicial de SWITCH y EVEN para la implementación en coma flotante.

En relación con el comportamiento estadístico, nuestros resultados muestran que SWITCH tiene una mejora marginal en la mezcla con respecto a LOG (y TENT, por supuesto).
Sin embargo, la mayor mejora se produce cuando se aplica el skipping, podemos ver que las entropías de BP y BPW crecen y las complejidades de BP y BPW disminuyen, para una dada representación numérica.
Este resultado es relevante porque evidencia de que un período largo no es sinónimo de buenas estadísticas, los mapas con skipping EVEN y ODD tienen longitudes de período de la mitad que los de SWITCH, pero su mezcla es mejor y sus distribuciones de amplitud se mantienen casi iguales.
Como contrapartida, se necesita más precisión para alcanzar las mejores asíntotas que ofrecen el método de skipping.

Resultó muy interesante el hecho de que le mapa TENT con $u=2$ (el cual produce salidas que convergen rápidamente a cero) y $u=1.96$ (con propiedades estadísticas mejores que LOG) produzcan salidas con los mismos resultados en el esquema switcheado.