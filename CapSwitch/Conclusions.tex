\section{Conclusiones}

Se exploró la degradación estadística debido al error inherente de los sistemas en base dos para los mapas caóticos simples, conmutados y con skipping.
Se evaluaron las distribuciones de mezcla y amplitud desde un punto de vista estadístico.

Este trabajo complementa los resultados anteriores dados en \cite{Nagaraj2008}, donde se investigaron las longitudes de los períodos.
En ese sentido, los resultados obtenidos aquí fueron compatibles.
Es posible observar que la conmutación entre dos mapas aumenta la dependencia del período en función de la precisión, esto se debe a que la longitud de correlación también se incrementa.
Sin embargo, el procedimiento estándar de skipping reduce la duración del período a casi la mitad.

Todas las estadísticas de los mapas representados en punto fijo producen una evolución no monótona hacia los resultados de punto flotante.
Este resultado es relevante ya que muestra que no siempre es conveniente aumentar la precisión empleada.

Es especialmente interesante observar que algunos sistemas (TENT) con muy buenas propiedades estadísticas en el mundo de los números reales, se vuelven ``patológicos'' cuando se usan representaciones numéricas binarias.
Como regla general, si un mapa se genera sólo por operaciones de desplazamiento (esto depende de la base de la unidad lógica aritmética y del mapa en sí), todas las condiciones iniciales convergerán a un punto fijo con un transitorio no mayor que la longitud de la mantisa que está siendo utilizada.

Al comparar los cuantificadores BP y BPW, se pudo detectar caídas a puntos fijos y se pudo estimar la longitud relativa de sus transitorios.
Esto puede verse en todas las implementaciones del mapa TENT y en una condición inicial de los mapas SWITCH y EVEN para la implementación en punto flotante.

En relación con el comportamiento estadístico, los resultados obtenidos muestran que el mapa SWITCH es levemente mojor en la mezcla con respecto al mapa LOG (y TENT, por supuesto).
Sin embargo, la mayor mejora se obtiene cuando se aplica el procedimiento de skipping, es posible ver que las entropías de BP y BPW crecen y las complejidades de BP y BPW disminuyen, para una dada representación numérica.
Este resultado es relevante ya que evidencia de que un período largo no es sinónimo de buenas propiedades estadísticas, los mapas con skipping EVEN y ODD tienen longitudes de período de la mitad que los de SWITCH, pero su mezcla es mejor y sus distribuciones de amplitud se mantienen casi iguales.
Como contrapartida, se necesita más precisión para alcanzar estas mejoras que ofrece el método de skipping.

Resultó muy interesante el hecho de que le mapa TENT con $u=2$ (el cual produce salidas que convergen rápidamente a cero) y $u=1.96$ (con propiedades estadísticas mejores que LOG) produzcan salidas con los mismos resultados en el esquema switcheado.