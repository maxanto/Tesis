%Plantilla basada en "Template for Masters / Doctoral Thesis" (plantilla disponible en writeLaTex) que subió LaTeXTemplates.com

\documentclass[a4paper]{book}
\usepackage[paperwidth=17cm, paperheight=22.5cm, bottom=2.5cm, right=2.5cm]{geometry}
\usepackage{amssymb,amsmath,amsthm} %paquete para símbolo matemáticos
\usepackage[spanish]{babel}
\usepackage[utf8]{inputenc} %Paquete para escribir acentos y otros símbolos directamente
\usepackage{enumerate}
\usepackage{graphicx}
%\usepackage{subfig} %para poner subfiguras
\graphicspath{{Img/}} %En qué carpeta están las imágenes
\usepackage[nottoc]{tocbibind}
\usepackage[pdftex,
            pdfauthor={NOMBRE DEL AUTOR},
            pdftitle={TÍTULO DE LA TESIS},
            pdfsubject={ÁREA DE LA TESIS},
            pdfkeywords={PALABRAS CLAVE},
            pdfproducer={Latex con hyperref},
            pdfcreator={pdflatex}]{hyperref}



\begin{document}

%----------------------------------------------------------------------------------------
%	COMANDOS PERSONALIZADOS
%----------------------------------------------------------------------------------------

%SI TU TESIS TIENE TEOREMAS Y DEMOSTRACIONES, PUEDES DESCOMENTAR Y USAR LOS SIGUIENTES COMANDOS

%\renewcommand{\proofname}{Demostración}
%\providecommand{\norm}[1]{\lVert#1\rVert} %Provee el comando para producir una norma.
%\providecommand{\innp}[1]{\langle#1\rangle} 
%\newcommand{\seno}{\mathrm{sen}}
%\newcommand{\diff}{\mathrm{d}}

%\newtheorem{teo}{Teorema}[section] 
%\newtheorem{cor}[teo]{Corolario}
%\newtheorem{lem}[teo]{Lema}

%\theoremstyle{definition}
%\newtheorem{dfn}[teo]{Definición}

%\theoremstyle{remark}
%\newtheorem{obs}[teo]{Observación}

%\allowdisplaybreaks



%----------------------------------------------------------------------------------------
%	PORTADA
%----------------------------------------------------------------------------------------

\title{TesisMaxi} %Con este nombre se guardará el proyecto en writeLaTex

\begin{titlepage}
\begin{center}

\textsc{\Large Facultad de Inveniería \- Universidad Nacional de Mar del Plata}\\[4em]

\vspace{4em}

\textsc{\huge \textbf{Sistemas Complejos, Ruidos Discretos y su implementación en FPGA}}\\[4em]

\textsc{\large Tesis}\\[1em]

\textsc{para obtener el título de}\\[1em]

\textsc{Doctor en Ingeniería con Orientación en Electrónica}\\[1em]

%\textsc{presenta}\\[1em]

\textsc{\Large Maximiliano Antonelli}\\[1em]

%\textsc{\large Asesor: NOMBRE}

\end{center}

\vspace*{\fill}
\textsc{Mar del Plata, Argentina \hspace*{\fill} 2016}

\end{titlepage}

%----------------------------------------------------------------------------------------
%	DEDICATORIA
%----------------------------------------------------------------------------------------

\pagestyle{empty}
\frontmatter

\chapter*{}
\begin{flushright}
\textit{A Sonia y Eduardo, que hicieron la mejor versión de mí que pudieron.\\
A Lorena, que sigue intentándolo.\\
A Giuliana y Luca, que les sale sin querer.}
\end{flushright}

%----------------------------------------------------------------------------------------
%	AGRADECIMIENTOS
%----------------------------------------------------------------------------------------

\chapter*{Agradecimientos}
%\markboth{AGRADECIMIENTOS23}{AGRADECIMIENTOS} % encabezado 

¡Muchas gracias a todos!

\tableofcontents


%----------------------------------------------------------------------------------------
%	TESIS
%----------------------------------------------------------------------------------------
\mainmatter %empieza la numeración de las páginas
\pagestyle{headings}

%  Incluye los capítulos en el folder de capítulos

En los últimos años se ha establecido que existen sistemas deterministas que rompen con el preconcepto de que los sistemas físicos pueden clasificarse en dos conjuntos disjuntos: sistemas deterministas y sistemas estocásticos.
El concepto antiguo era que un sistema determinista es aquél para el cual conocemos el modelo y por lo tanto es posible predecir con exactitud la evolución de sus variables de estado.
Se utilizan en su descripción ecuaciones diferenciales o de recurrencia.
Por otra parte un sistema estocástico es aquél para el cual el modelo no se conoce o se lo supone sumamente complejo como para ser obtenido, de modo que se adopta la estrategia de estudiar sus variables de estado en forma estadística.
Se utilizan entonces en la descripción ecuaciones diferenciales o de recurrencia estocásticas.

El caos determinista demostró que complejidad en la evolución temporal no es sinónimo de complejidad en el modelo, cuando hay alinealidad: modelos deterministas muy simples originan señales de aspecto estocástico.
La sensibilidad a las condiciones iniciales hace que en estos sistemas la predictibilidad sea a corto plazo (luego de un tiempo finito es imposible predecir la evolución) lo que ubica a estos sistemas en una posición intermedia entre determinista y estocástico \cite{Liao2013a}.

Como consecuencia se desarrollaron en los últimos años un número creciente de aplicaciones de los sistemas caóticos, empleándolos principalmente como generadores de ruido controlado \cite{DeMicco2007C}, generadores de números pseudoaleatorios \cite{DeMicco2007A}, portadoras de señales \cite{DeMicco2007B}, sistemas de encriptado \cite{Machado2004, Smaoui2009}, etc.

Hoy en día, los sistemas dinámicos son un objeto de estudio interdisciplinario, aunque originalmente fue una rama de la física.
Todo comenzó a mediados del 1600, cuando Newton inventó las ecuaciones diferenciales, descubriendo sus leyes del movimiento de gravitación universal, y las combinó con las leyes de Kepler sobre el movimiento planetario.
Específicamente, Newton resolvió el problema de los dos cuerpos (por ejemplo, el sistema tierra-sol).

Subsecuentes generaciones de matemáticos y físicos intentaron extender los métodos analíticos de Newton al problema de los tres cuerpos (por ejemplo, luna-tierra-sol), pero curiosamente para resolver este problema se necesitó mucho más esfuerzo.
Luego de décadas, se dieron cuenta de que el problema de los tres cuerpos era esencialmente imposible de resolver, en el sentido de obtener las fórmulas explícitas.

La ruptura vino con el trabajo de Poincaré a finales del 1800.
Él introdujo un nuevo punto de vista que enfatizaba las cuestiones cualitativas más que las cuantitativas (por ejemplo, ¿es estable el sistema luna-tierra-sol?).
Poincaré desarrolló una poderosa aproximación geométrica que hoy es usada para estudiar sistemas dinámicos y también fue el primero en vislumbrar la posibilidad del caos, en el cual un sistema determinístico exhibe un comportamiento aperiódico que depende sensiblemente de las condiciones iniciales, haciendo así imposible la predicción a largo plazo.

Pero el caos se mantuvo en segundo plano hasta la segunda mitad del 1900, en donde los osciladores no lineales jugaron un rol vital en el desarrollo de tecnologías de radio, radar, lazos de enganche de fase y láser.
Por el lado matemático, los osciladores no lineales también estimularon la invención de nuevas técnicas matemáticas.
Los métodos geométricos de Poincaré se fueron extendiendo para producir un conocimiento mucho más profundo de la mecánica clásica.

La invención de la computadora por el 1950 fue una línea divisoria en la historia de los sistemas dinámicos.
La computadora nos permite experimentar con ecuaciones en una forma que antes era imposible, y así explorar la dinámica los sistemas no lineales de una forma mucho más directa.
Estos experimentos llevaron a Lorenz a descubrir en 1963 el movimiento caótico de un atractor extraño, mientras estudiaba un modelo simplificado de la circulación de convección para comprender mejor la notoria impredictibilidad del clima.
Lorenz encontró que la solución a sus ecuaciones nunca caían al equilibrio o a un estado periódico.
Además, si comenzaba sus simulaciones de dos condiciones iniciales ligeramente diferentes, los comportamientos resultantes pronto serían
totalmente diferentes.
Como consecuencia de ello, el sistema es inherentemente impredecible, pequeños errores en las mediciones del estado actual de la atmósfera (o cualquier sistema caótico) sería amplificado rápidamente.
Pero Lorenz también mostró que había estructura en el caos, cuando las soluciones fueron dibujadas en tres dimensiones, las soluciones a sus ecuaciones cayeron sobre un conjunto de puntos en forma de mariposa.
Él sostuvo que este sistema tenía que ser “un infinito complejo de superficies”, lo que hoy podríamos considerar como un ejemplo de fractal.

El trabajo de Lorenz tuvo un pequeño impacto hasta 1970, los años del boom del caos.
Se desarrollaron teorías completamente nuevas basadas en consideraciones sobre atractores caóticos, como turbulencia de fluidos y biología de las poblaciones y se encontraron comportamientos caóticos en reacciones químicas \cite{Kapral1995}, osciladores mecánicos \cite{Awrejcewicz2003}, semiconductores \cite{Scholl2001} y oscilaciones biológicas como el ritmo cardíaco y circadiano \cite{Strogatz2018}.
Hoy, la teoría del caos es una herramienta más para el estudio de sistemas dinámicos y los sistemas caóticos son utilizados en una gran cantidad de dispositivos.

En este Capítulo se revisan los conceptos de espacio de fases, pasando por las soluciones típicas de sistemas de ecuaciones diferenciales, para luego poder entrar a la descripción de sistemas caóticos. Primero se abordan los sistemas caóticos continuos con derivada continua y presentamos tres ejemplos clásicos en la literatura. Luego se hace una reseña a los mapas caóticos, en donde presentamos los mapas cuadráticos bidimensionales, los cuales usaremos en algunas secciones subsiguientes.

\thispagestyle{empty}

\include{Gen}
\thispagestyle{empty}

\include{Cuanti}
\thispagestyle{empty}

\include{Arit}
\thispagestyle{empty}

\section{Conclusiones}

En este capítulo presentamos las principales herramientas utilizadas para detectar caos y cuantificar la calidad estadística de los generadores de números aleatorios.
Junto con la introducción tórica, se mostraron algunos avances en la implementación de dichas herramientas.

El algoritmo evolutivo desarrollado detecta con precisión el máximo $ MLE $ del sistema en cada región en el espacio de parámetros del conocido oscilador Logístico.
El siguiente paso es reemplazar el oscilador logístico por el sistema de multiatractores caótico descrito en la sección \ ref {caos}.
La búsqueda exhaustiva de $ MLE $ barriendo todos los valores de parámetros se vuelve muy complicada cuando aumenta el número de parámetros.
Esta es la razón por la cual se empleó un algoritmo genético en este trabajo.
Este algoritmo heurístico permite encontrar las áreas de interés, p. $ MLE> 0 $, de una manera más rápida y simple.
Hoy en día, estamos trabajando para finalizar la implementación de hardware de todo el sistema.
En la implementación de hardware del cálculo $ MLE $, hemos explotado la naturaleza paralela de subrayado de las ecuaciones de cálculo $ MLE $ con el objetivo de optimizar el diseño de arquitectura propuesto, permitiendo su implementación concurrente basada en tecnología FPGA.

Se desarrolló e implementó un sistema que permite medir con buena precisión las entropías causal y no-causal de señales analógicas provenientes del exterior de la FPGA y también internas generadas por código.
Se logró medir señales y realizar cálculos complejos con un microcontrolador modesto como el 8051 instanciado en la FPGA AFS1500 de ACTEL.
Este primer prototipo cumple con las especificaciones de precisión y cantidad de recursos requeridos establecidas en el diseño, el próximo paso será optimizar el sistema en cuanto a frecuencia de operación e inmunidad al ruido.
Se prevé que el sistema permita modificar, en tiempo de ejecución, la frecuencia de muestreo, de forma de que sea adaptable a la señal de entrada, con el límite superior de 500~Ks/s fijado por el ADC.
Deberá agregarse también un umbral a partir del cual un valor es considerado distinto de otro, de esta forma se solucionaría el problema que presenta el ruido aditivo en el cálculo de $H_{BP}$.
El código de este sistema ocupa el 15,4\% del total de la memoria flash del micro instanciado, por lo que será posible agregar \textit{software} para implementar otros cuantificadores y funcionalidades.
En cuanto a los recursos disponibles en la FPGA se utilizaron 7349 celdas lógicas, quedando casi el 80\% de los recursos de \textit{hardware} disponibles para implementar los sistemas bajo prueba en forma concurrente.

También se exploraron las fuentes de error en un medidor de entropías implementado en FPGA.
Para este primer análisis evaluamos que sucede aplicando un filtro abrupto, es por esto que elegimos para comparar un filtro elíptico y uno ideal. 
Las respuestas del filtro elíptico y del ideal fueron muy similares en el rango de frecuencias en los que el elíptico tiene un buen comportamiento, sin embargo cuando la frecuencia de corte del elíptico se acerca a los extremos (es decir cuando $f_c \to 0$ o $f_c \to 1$) la salida del filtro diverge.
El problema se debe a que el método numérico utilizado para calcular la salida del filtro diverge por la precisión finita utilizada.
Como no necesitamos volver a la frecuencia contínua nos quedamos con los resultados del ideal para hacer las pruebas, sin tener que procuparnos por el ripple que aparece en las bandas de paso y rechazo cuando pasamos al mundo analógico.
Cuando comparamos las respuestas de los cuantificadores con y sin ruido, vemos que las señales limpias tienen mesetas, es decir que se mantienen constantes hasta que el filtrado elimina la siguiente componente espectral.
Sin embargo, cuando son contaminadas con ruido los cuantificadores cambian para parecerse más a los resultados que arroja el ruido blanco gaussiano sin ninguna señal determinística.
En todos los casos se vio que estos cuantificadores son muy sensibles a la presencia de ruido, los que nos permite vincular a este hecho los errores en la medición.
También vimos que los valores cambian a medida que se filtra la señal sin contaminar, lo que agrega una segunda fuente de error dada por el ancho de banda finito del sistema.
Para continuar con este proyecto faltaría, por un lado caracterizar el sistema de medición en cuanto a su ancho de banda y su rechazo al ruido aditivo, y por otro lado probar con otros cuantificadores (como complejidad, desequilibrio, entropía diferencial, rate entropy, etc) o con variantes de los presentados aquí (Bandt \& Pompe pesada, amplitud promedio en el emmbedding, etc).
\thispagestyle{empty}

%----------------------------------------------------------------------------------------
%	APÉNDICES
%----------------------------------------------------------------------------------------

\addtocontents{toc}{\vspace{2em}} % Agrega espacios en la toc

\appendix % Los siguientes capítulos son apéndices

%  Incluye los apéndices en el folder de apéndices

\chapter{Field Programmable Gate Array (FPGA)}

Cosas que distraen en la tesis.
\thispagestyle{empty}
%\include{Apendices/AppendixB}
%\include{Apendices/AppendixC}

\addtocontents{toc}{\vspace{2em}} % Agrega espacio en la toc


%----------------------------------------------------------------------------------------
%	BIBLIOGRAFÍA
%----------------------------------------------------------------------------------------
\backmatter
\nocite{*}
\bibliographystyle{plain}
\bibliography{bibliografía.bib} %Aquí ponen el nombre del archivo .bib

\end{document}