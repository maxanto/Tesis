\section{Conclusiones}

El \textit{jitter} es inevitable en {RO}s, y en consecuencia, necesita ser caracterizado.
La mezcla y la distribución de valores son las principales propiedades a considerar.
Varios {ITQ} fueron evaluados aquí.
$S_W$, $S^{(D)}_{BP}$, $H_W$ y $H^{(D)}_{BP}$ resultan dependientes de los parámetros $W$ y $D$.
Esto es un inconveniente si se deben emplear como cuantificadores de aleatoriedad, ya que dependen de parámetros externos a la fuente generadora de símbolos.
Por otro lado, no es posible calcular \emph{rate entropies}, $h_0^*$ y $h_0$, ya que se necesita una cantidad infinita de datos para su cálculo.
Las dos {entropías diferenciales}, $h^*$ y $h$, en cambio, son independientes de los parámetros utilizados para su determinación y son estimadores de la \emph{rate entropy}.
Se demostró en la Sección \ref{sec:resu} que, en el caso de {RO}s muestreados, presentan un mínimo para la tasa de muestreo correcta, lo que los convierte en una buena medida de la calidad tanto de los {RO}s como de los {PRNG}s derivados de ellos.

El plano de entropía dual determinado por estos cuantificadores ha demostrado discernir satisfactoriamente entre las dos principales propiedades deseadas de {PRNG}, la equiprobabilidad entre todos los valores posibles y la independencia estadística entre valores consecutivos.
Por lo tanto, permite ver claramente lo que debe mejorarse en una secuencia determinada para obtener una buena \textit{PRNG}.
Los ejemplos presentados aquí han demostrado la necesidad de utilizar ambos histogramas para caracterizar secuencias.

En cuanto a la implementación en hardware, los {TRNG}s basados en {RO} implementados aquí han demostrado satisfacer las propiedades estadísticas deseadas para un {RNG}.
Son comparables a otros {RNG}s utilizados y en algunos casos presentan mejores características.
Emplean pocos recursos del dispositivo y se implementan de forma muy simple en una plataforma digital.

Se demostró que para esta arquitectura la cantidad de {RO}s establece las propiedades estadísticas del {TRNG}.
Se vio que para $15$ {RO}s el histograma y la mezcla, eran casi ideales, haciendo innecesario el aumento de la cantidad de anillos.