\subsection{Conclusiones}

Dada su utilidad como \emph{PRNG} y generadores de reloj, los \emph{RO}s se están convirtiendo en uno de los principales componentes básicos de los circuitos digitales.
El jitter es inevitable en \emph{RO}s, y en consecuencia, necesita ser caracterizado.
La mezcla y la distribución de valores son las principales propiedades a considerar.
Varios \emph{ITQ} fueron evaluados aquí.
$S_W$, $S^{(D)}_{BP}$, $H_W$ y $H^{(D)}_{BP}$ resultan dependientes de los parámetros $W$ y $D$.
Esto es un inconveniente si los usamos como medidas de inestabilidad.
Por otro lado, no es posible calcular \emph{rate entropies}, $h_0^*$ y $h_0$, ya que se necesita una cantidad infinita de datos para su cálculo.
Las dos \emph{entropías diferenciales}, $h^*$ y $h$, en cambio, son independientes de los parámetros utilizados para su determinación y son estimadores de la \emph{rate entropy}.
Hemos mostrado en la sección \ref{sec:resu} que en el caso de \emph{RO}s muestreados, presentan un mínimo para la tasa de muestreo correcta, lo que los convierte en una buena medida de la calidad tanto de los \emph{RO}s como de los \emph{PRNG}s derivados de ellos.

El plano de entropía dual determinado por estos cuantificadores ha demostrado discernir satisfactoriamente entre las dos principales propiedades deseadas de \emph{PRNG}, la equiprobabilidad entre todos los valores posibles y la independencia estadística entre valores consecutivos.
Por lo tanto, permite ver claramente lo que debe mejorarse en una secuencia determinada.
Los ejemplos presentados aquí han demostrado la necesidad de utilizar ambos histogramas para caracterizar secuencias.