\section{Conclusions}
\label{sec:conclu}
Given their usefulness as \emph{PRNG} and clock generators, \emph{RO}s are becoming one of the main building blocks of digital circuits. Jitter is unavoidable in \emph{RO}s, and consequently, it needs to be characterized. Mixing and distribution of values are the main properties to consider. 
Several \emph{ITQ} quantifiers were evaluated here. $S_W$, $S^{(D)}_{BP}$, $H_W$ and $H^{(D)}_{BP}$ turn out to be dependent on parameters $W$ and $D$. This is a drawback if we use them as jitter measures. On the other hand, it is no possible to calculate  \emph{rate entropies}, $h_0^*$ and $h_0$,  since an infinite number of data is necessary for their calculation. The two \emph{differential entropies}, $h^*$ and $h$, instead, are independent of the parameters used for their determination and are estimators of the \emph{rate entropies}. We have shown in Section \ref{sec:resu} that in the case of sampled \emph{RO}s they also present a minimum for the correct sampling ratio making them a good measure of the quality of both \emph{RO}'s and \emph{PRNG}'s derived from them. 

The dual entropy plane determined by these quantifiers has demonstrated to satisfactorily discern between the \emph{PRNG}'s two main desired properties, the equiprobability among all possible values and the statistical independence between consecutive values. Thus, it allows clearly seeing what needs to be improved in a given sequence.
The examples presented here have demonstrated the need to use both histograms for characterizing sequences.


