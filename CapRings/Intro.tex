\section{Introducción}
\label{sec:intro}

El \emph{Jitter} es cualquier desviación leve del período medio de una señal presuntamente periódica.
Hay muchos ejemplos físicos donde esta inestabilidad es relevante.
Algunos ejemplos de diferentes áreas son:
(a) Stalberg \textit{et al.} \cite{Stalberg1971} encontraron que el intervalo de tiempo entre los dos potenciales de acción de dos fibras musculares, que pertenecen a la misma unidad motora en los músculos humanos normales, muestra una variabilidad o inestabilidad;
(b) Mecozzi \textit{et al.} \cite{Mecozzi2001} detectaron jitter temporal y variaciones de amplitud en enlaces ópticos utilizando transmisión de pulso altamente disperso;
(c) Derickson \textit{et al.} \cite{Derickson1991} realizaron una comparación completa de la fluctuación de tiempo en el caso de los láseres semiconductores en modo bloqueado;
(d) el California y Carnegie Planet Search en el Observatorio Keck \cite{Wright2005} informó la inestabilidad de las estrellas en las velocidades radiales;
(e) Roberts \& Guillemin estudiaron los retardos debidos a las colas en etapas de \textit{upstream multiplexing}, en una red de Modo de transferencia asíncrono (ATM);
(f) Baron \textit{et al.} \cite{Baron2012} consideraron la calidad de la señal del \textit{bunch clock} del \textit{Large Hadron Collider} (LHC), en términos de inestabilidad, un problema fundamental porque sincroniza todos los sistemas electrónicos en el detector;
(g) Marsalek \textit{et al.} analizaron la relación entre la entrada sináptica y la fluctuación de fase de salida pico en neuronas individuales \cite{Marsalek1997}, etc.

Además, los instrumentos digitales se utilizan en cualquier experimento moderno y la inestabilidad inevitable en los sistemas de adquisición de datos produce incertidumbres en el tiempo y, por consiguiente, en cualquier determinación del espectro.

En esta aplicación particular de los ROs, el \textit{jitter} no siempre es indeseable.
El \textit{jitter} no es deseado en aplicaciones que usan un RO como generador de reloj \cite{Buedo1998, Beomsup1990, Hajimiri1999, Mandal2010, Gupta2011}.
Por el contrario, los generadores de números aleatorios RNG basados en \emph{RO}s, usan el \textit{jitter} como fuente de aleatoriedad, \cite{Sunar2007, Wold2009}.
El \textit{jitter} también puede ser utilizado para mejorar la compatibilidad electromagnética (EMC) en un circuito digital para distribuir la frecuencia del reloj sobre una banda \cite{DeMicco2012}.

La determinación del \textit{jitter} de fase en ROs se ha estudiado en varios artículos: en \cite{McNeill1997} se presentó el estudio de tres medidas relevantes del \textit{jitter} en el dominio del tiempo.
En \cite{Valtchanov2008} se propuso un modelo para la generación y distribución del \textit{jitter} en ROs.
En este artículo, los autores separan las fuentes de inestabilidad en deterministas y aleatorias (gaussianas); además, cada fuente se clasifica adicionalmente en local o global.
Demuestran que las contribuciones más importantes son la inestabilidad gaussiana local y la inestabilidad determinística global y sólo la primera debe usarse como una fuente de aleatoriedad de generadores de TRNG.
El mismo enfoque se usó en \cite{Fischer2008, Valtchanov2010, Baudet2011, Jessa2011}.
En \cite{Lubicz2014} Lubicz \textit{et al.} describen un método práctico y eficiente para estimar la tasa de entropía de un \emph{TRNG} basado en osciladores libres; enfatizaron que su método no requiere extraer las señales del dispositivo y analizarlas con equipos externos (una metodología que introduce fluctuación y distorsión extra en la señal medida debido a la cadena de adquisición de datos).

Por lo general, \emph{jitter determinista} es el nombre que se le da a cualquier \emph{jitter no gaussiano}.
Este \emph{jitter} está limitado y se caracteriza por su valor máximo de $\Delta_{pp}$.
\textit{Jitter} aleatorio es el nombre utilizado para la \textit{jitter} gaussiano y se caracteriza por su valor RMS.
También frecuentemente las señales presentan \textit{jitter} periódico determinístico, en este caso el tiempo entre flancos consecutivos se incrementa y decrementa a intervalos regulares.
Estos intervalos entre dos tiempos el efecto máximo son el período del \textit{jitter}; el inverso de este período  es la frecuencia del jitter.
El \textit{jitter} periódico con frecuencia de fluctuación inferior a $10 Hz$ usualmente se denomina \emph{wander} y el nombre \emph{jitter} se refiere sólo a las fluctuaciones periódicas con frecuencias en o por encima de $10 Hz$.
En comunicaciones, \emph{jitter total} es $T = \Delta_{pp} + 2nR_{rms}$ donde $n$ es un número entre $6$ y $8$ relacionado con la tasa de error binsrio (\emph{BER}).

Los \emph{RO}s son uno de los principales componentes de los circuitos integrados analógicos y digitales y se han utilizado ampliamente como osciladores \emph{on-chip} para generar relojes en circuitos de alta velocidad.
Además, los \emph{RO}s se pueden implementar fácilmente en circuitos digitales programables como \emph{FPGA}s.
Las principales ventajas de los osciladores integrados \emph{RO} sobre los clásicos circuitos resonantes Bobina-Capacitor son su área de chip más pequeña, su rango de funcionamiento más amplio (que puede ser sintonizado eléctricamente) y su menor consumo de energía.

Ya sea que se quiera usarlo o eliminarlo, el \textit{jitter} en \emph{RO}s debe medirse, lo que no es una tarea simple.
La principal contribución de este trabajo es proporcionar una técnica de medición del \textit{jitter} basada en cuantificadores de la teoría de la información (\emph{ITQ}).
Se utilizó un modelo estocástico cuya aleatoriedad está relacionada con la amplitud de la inestabilidad.
Cada \emph{ITQ} propuesto utilizado en este trabajo se basa en una entropía, es decir, una función de Shannon de la \emph{PDF} asignada a la serie de tiempo del proceso estocástico.
También se pueden usar desequilibrios y complejidades \cite{Amigo2005,Rosso2007B}, pero no representan una mejora en este caso.
En este caso utilizamos dos opciones para \emph{PDF}: el \emph{histograma normalizado} y el \emph{histograma de patrones de orden}.
Se usa un plano de representación para comparar diferentes situaciones.
Una vez que se elige la \emph{PDF}, la Entropía de Shannon es la función básica que cuantifica la uniformidad de la \emph{PDF}.
Las \emph{entropías normalizadas}, \emph{entropías diferenciales} y \emph{tasa entropía} son las otras \emph{ITQ}s evaluadas.
En este caso las \emph{entropías diferenciales} obtienen los mejores resultados y se utiliza un \emph{plano de entropías diferenciales} para comparar su sensibilidad como medida de \textit{jitter}.