\section{Determinación del \textit{jitter} en \emph{RO}'s}
\label{sec:jitter}

Hay dos situaciones diferentes en lo que concierne al \textit{jitter} en \textit{RO}s:
(a) en algunas aplicaciones es suficiente con asegurar que el \textit{jitter} no perturba a la señal por encima de un límite aceptable.
En este caso la señal se observa en un osciloscopio con un amáscara sobre la pantalla, lo que es suficiente para verificar que la señal se mantiene dentro de los márgenes de tolerancia;
(b) en otros casos se precisa una determinación exacta del \textit{jitter}.
Entre esos casos está la caractarización de \textit{RO}s considerada en este trabajo.

Los \textit{RO}s ideales están compuestos por un numero impar de inversores.
Cada inversor tiene un tiempo de propagación y por lo tanto los flancos de subida y bajada separados por medio período viajan a través de los inversores.
Si todos los tiempos de propagación son constantes, la salida de este \textit{RO} ideal es una señal cuadrada con un espectro de frecuencia discreto.
Pero los tiempos de propagación no son constantes, por lo tanto hay \textit{jitter}.
El \textit{jitter} distorsiona el espectro de potencia ensanchando cada delta en un máximo con cierta anchura.

Supongamos que $T/2$ es el medio período de un \emph{RO} ideal.
Entonces está dado por:
%
\begin{eqnarray}
\frac{T}{2}=k \sum_{i=1}^{k}d_i
\end{eqnarray}
%
En donde $k$ es el número de inversores y $d_i$ es el tiempo de propagación a través del $i-$ésimo inversor.
Cuando hay jitter, $d_i$ es una variable aleatoria que modelamos como:
%
\begin{eqnarray}
d_i=D_i+ \Delta d_i
\end{eqnarray}
%
donde $D_i$ es el valor medio de $d_i$ con el nivel nominal de voltaje de fuente y la temperatura normal, y $\Delta d_i$ es la variación del retardo producida por los eventos físicos locales y los cambios globales en las condiciones de trabajo del dispositivo (como $V_{CC}$, temperatura , etc.).
Entonces, el \textit{jitter} en \emph{RO}s se evidencia por el desplazamiento aleatorio de la ubicación de los flancos ascendentes (descendentes), con respecto a la ubicación perfectamente periódica.\
La medición directa de este desplazamiento tiene dos problemas principales:
(a) requiere un instrumento de muy alta frecuencia, porque la resolución del tiempo está limitada por el período de muestreo $T_s$;
(b) esta técnica introduce fluctuaciones y distorsiones adicionales en la señal medida proveniente de la cadena de adquisición de datos.
Entonces es más conveniente usar \emph{medidas indirectas}, por medio de variables aleatorias auxiliares relacionadas con las propiedades estadísticas relacionadas con el \textit{jitter} para medir la fluctuación de fase con una perturbación mínima \cite{Lubicz2014}.
El procedimiento general es el siguiente:
\begin{enumerate}
\item Se muestrea la salida con el período de muestreo $ T_s $ para obtener una serie de tiempo binaria.
En el caso ideal de \emph{no-jitter}, la salida es una \emph{onda cuadrada continua y perfectamente periódica} con un período $T$.
Entonces es posible ajustar $T_s$ para hacer $T/2 = mT_s $ con $m \in N^+$.
La serie de tiempo binaria será periódica con $m$ unos seguida de $m$ ceros.
Cuando el \textit{jitter} está presente, la serie binaria no es periódica sino estocástica.
Este modelo estocástico se conoce como \emph{proceso de renovación alterna}.
\item Se pueden usar muchos cuantificadores de aleatoriedad diferentes para caracterizar el modelo estocástico asociado con el \textit{jitter} medido.
En este trabajo utilizamos cuantificadores de la teoría de la información.
\end{enumerate}
Tengamos en cuenta que el \textit{jitter} es acumulativo y surgen dos situaciones básicas:
(a) si se supone que el \textit{jitter} introducido por cada etapa es totalmente independiente del \textit{jitter} introducido por otras etapas, significa que  $\sigma_T^2=m*\sigma_s^2$, en donde $\sigma_s$ es el \textit{jitter} de cada muestra, y se supone que todas las muestras tienen fluctuaciones con la misma distribución normal;
(b) si las fuentes de \textit{jitter} están totalmente correlacionadas entre sí, entonces $\sigma_T=m*\sigma_s$.