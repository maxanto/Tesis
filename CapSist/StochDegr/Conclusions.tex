\subsection{Conclusion} \label{sec:conclusions}

En este trabajo, hemos desarrollado un análisis detallado de los cambios en el comportamiento de una implementación en punto fijo de un mapa cuadrático bidimensional.
Nuestro objetivo es reportar la tasa de degradación de la propiedad de cada sistema, para que los autores la utilicen al momento de diseñar sus aplicaciones particulares.

Los resultados muestran que es posible determinar un umbral para el número de bits empleados en la representación de punto fijo del sistema, mientras que el dominio de atracción conserva su integridad y se mantienen las características de las secuencias generadas.
Con la ayuda de los cuantificadores de aleatoriedad introducidos fue posible determinar ese límite, en el caso del estudio fue de 23 bits.
El mismo procedimiento debe repetirse para cualquier otro sistema si se desea utilizarlo en una aplicación electrónica digital, como generadores de ruido controlados o para desarrollar nuevos sistemas de encriptación.
