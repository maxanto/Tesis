\subsection{Conclusion} \label{sec:conclusions}
In this work, we have developed a detailed analysis of the changes in behaviour of a $2D$-quadratic map fixed-point implementation.
Our goal is to report the rate of degradation for each systems' property, so as to be used by authors at the time of designing their particular applications.
Results show that it is possible to determine a threshold for the number of bits employed in the fixed point representation of the system, whereas the domain of attraction preserves its integrity and the characteristics of the generated sequences are kept.
With the help of the quantifiers of randomness introduced it was possible to determine that limit, in the case of study it was 23 bits.
The same procedure should be repeated for any other system if it is desired to be used in a digital electronic application, such as controlled noise generators or to develop novel encryption systems.
In the particular case of Sprott's chaotic system, if the minimum number of 23 bits is satisfied at the time of digitalizing, we conclude that it is possible to successfully use it as a component of new encryption algorithms.