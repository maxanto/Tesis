\subsubsection{Herramientas de análisis}
\label{sec:quanti}

La entropía de Shannon normalizada aplicada a dos PDFs diferentes y el máximo exponente de Lyapunov junto con las longitudes del período medio son los cuantificadores empleados aquí para estimar las propiedades del sistema.
Las entropías nos ayudan a evaluar las dos propiedades que determinan el grado de aleatoriedad, la equiprobabilidad entre todos los valores posibles y la independencia estadística entre valores consecutivos, mientras que el MLE determina la presencia de caos.
 
\subsubsection{Longitud de período}

Usando $n$ bits para representar las variables de estado de un sistema $D$ -dimensional, el período teórico máximo $T_{max}$ que se puede alcanzar es $T_{max} = 2 ^ {Dn}$.
En realidad, los períodos obtenidos son mucho más bajos que el máximo y dependen en gran medida de la IC.

Hemos desarrollado un algoritmo que emula el funcionamiento del sistema en un entorno digital.
Una tarea de este código es analizar el período alcanzado al iniciar la iteración a partir de cada condición inicial usando diferentes precisiones en una arquitectura de punto fijo.
Cada semilla podría converger a un ciclo límite, o podría ser un valor del ciclo límite en sí mismo.
Este procedimiento se repitió para todas las condiciones iniciales para obtener el esquema de dominio de atracción del sistema.

\subsubsection{Cuantificadores de aleatoriedad}
\label{cu_ran}

Los nuevos cuantificadores propuestos aquí nos brindan información sobre el grado de aleatoriedad, que no está disponible en las pruebas `` pasa - no pasa '' como las utilizadas como estándar para evaluar generadores de números aleatorios.

En base a los resultados de investigaciones anteriores \cite{DeMicco2008, Antonelli2016, DeMicco2011} se adoptó la entropía de Shannon normalizada como cuantificador para caracterizar el determinismo y la estocasticidad de las secuencias generadas.
