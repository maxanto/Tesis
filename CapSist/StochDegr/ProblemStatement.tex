\subsection{Conceptos Preliminares} \label{sec:estudio}

Al iterar mapas caóticos en $\mathbb{R} ^ 2 $, después de un transitorio que depende del parámetro de mezcla ($r_{mix}$), la secuencia generada limita en un punto o colección de puntos llamado atractor.
Un mapa caótico puede tener uno o más atractores.
Dominio de atracción se llama a todas las condiciones iniciales (IC) que convergen a cada atractor.
Las secuencias ergódicas de los atractores, generadas por el mapa, tienen una distribución determinada llamada Función de densidad de probabilidad invariable (IPDF).
Las principales características de los mapas caóticos, IPDF y $r_{mix}$, pueden obtenerse calculando el operador Perron-Frobenius (PFO), que depende de la estructura del mapa.
Los puntos fijos de su espectro son las densidades invariables y corresponden a los vectores propios con valor propio igual a uno, la constante de mezcla corresponde al segundo mayor valor propio del PFO, \cite{Lasota1994, Lasota1973}.

Cuando se utiliza precisión finita, este análisis no es válido, todos los atractores toman la forma de puntos fijos u órbitas periódicas.
El PFO del mapa ya no describe las características de las secuencias.
Con respecto al dominio de atracción, también cambiará cuando se digitalice, cada valor inicial será parte de, o convergerá a, un cierto punto fijo u órbita periódica.
En general, aparecen muchas nuevas órbitas periódicas y cambian cuando varía el número de bits empleados.

Con el propósito de utilizar estos sistemas en aplicaciones electrónicas, se hace necesario comprender cómo evoluciona el dominio de atracción con la variación de bits empleados.
Principalmente es importante saber cuál es la duración del período y el \textsl{grado de aleatoriedad} del ciclo en el que converge cada semilla.
Por esta razón, hemos incluido cuantificadores de aleatoriedad que estiman indirectamente el $r_{mix}$ e IPDF del sistema digitalizado.

In this paper we have emulated the behaviour of a digital hardware implementation, such as FPGA, Complex Programmable Logic Device (CLPD) or Application Specific Integrated Circuit (ASIC), to exactly replicate the operation of the device.
Our interest is to measure how the domains of attraction degrade with a change in the number of bits $n$ employed, as well as to find the threshold value 
$n_{min}$. 

En este documento hemos emulado el comportamiento de una implementación en hardware digital, como FPGA, Dispositivo lógico programable complejo (CLPD) o Circuito integrado de aplicacipnes específicas (ASIC), para replicar exactamente el funcionamiento del dispositivo.
Nuestro interés es medir cómo los dominios de atracción se degradan con un cambio en el número de bits $n$ empleados, así como también encontrar el valor umbral
$n_{min}$.
