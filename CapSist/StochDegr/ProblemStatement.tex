\subsection{Preliminary Concepts} \label{sec:estudio}
When iterating chaotic maps in $\mathbb{R}^2$, after a transient that depends on the mixing parameter ($r_{mix}$), the generated sequence limits in a point or a collection of points called an attractor.
A chaotic map can have one or more attractors.
Attractor domain is called to all the initial conditions (ICs) that converge to each attractor.
The ergodic sequences of the attractors, generated by the map, have a determined distribution called Invariant Probability Density Function (IPDF).
Main characteristics of chaotic maps, IPDF and $r_{mix}$, can be obtained by calculating the Frobenius-Perron operator (FPO), which depends on the map's structure.
The fixed points of its spectrum are the invariant densities and they correspond to the eigenvectors with eigenvalue equal to one, the mixing constant corresponds to the second largest eigenvalue of the FPO, \cite{Lasota1994,Lasota1973}.

When using finite precision, this analysis is not valid, all attractors take the form of fixed points or periodic orbits.
The FPO of the map no longer describes the sequences' characteristics.
Regarding the attractor domain, it will also change when digitalized, each initial value will be part of, or will converge to, a certain fixed point or periodic orbit.
Generally, many new periodic orbits appear, and change when the number of bits employed varies.

With the purpose of utilizing these systems in electronic applications it becomes necessary to understand how the attraction domain evolves with the variation of bits employed.
It is mainly important to know which is the period's length and the \textsl{randomness degree} of the cycle at which each seed converges.
For this reason, we have included randomness quantifiers that indirectly estimate a sort of $r_{mix}$ and IPDF of the digitalized system.

In this paper we have emulated the behaviour of a digital hardware implementation, such as FPGA, Complex Programmable Logic Device (CLPD) or Application Specific Integrated Circuit (ASIC), to exactly replicate the operation of the device.
Our interest is to measure how the domains of attraction degrade with a change in the number of bits $n$ employed, as well as to find the threshold value 
$n_{min}$. 
