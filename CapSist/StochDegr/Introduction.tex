\subsection{Introducción} \label{sec:intro}

Los sistemas caóticos tienen un número cada vez mayor de aplicaciones y su implementación se debe principalmente a la extrema sensibilidad a las condiciones iniciales.
En general, estos sistemas se utilizan para la generación de ruidos controlados, estos generadores de ruido pseudoaleatorios digitales (PRNG) se pueden emplear en un gran número de aplicaciones electrónicas, como secuencias de cifrado para privacidad, técnicas de multiplexación, compatibilidad electromagnética \cite{Machado2004, Smaoui2009, DeMicco2007A, DeMicco2007C, DeMicco2007B}.
En computadoras y dispositivos digitales, solo se pueden generar atractores \textsl{pseudocaóticos}.
Pero la discretización puede destruir el comportamiento \textsl{pseudocaótico} y, en consecuencia, es un proceso no trivial.

Se han propuesto varias estrategias para una selección correcta del número óptimo de bits en las implementaciones de hardware.
Sin embargo, la mayoría de estos procedimientos están limitados a sistemas lineales \cite{Constantinides2002, Constantinides2003}.
En los sistemas digitales caóticos, se puede obtener un comportamiento completamente diferente al variar la precisión.
Este problema ha ganado interés recientemente, y se han propuesto varios esquemas nuevos \cite{Ding2007, Asseri2002, Azzaz2009}.

En resumen, a pesar de la aritmética utilizada, ya sea de punto fijo o punto flotante, el conjunto de números que se pueden representar es limitado.
Incluso utilizando una precisión extremadamente alta como lo hacen Liao y Wang \cite{Liao2013a}, las secuencias generadas por un sistema caótico que usa hardware digital siempre serán periódicas.

El trabajo de Grebogi \cite{Grebogi1988} mostró que la longitud promedio de las órbitas periódicas $T$ de un sistema dinámico implementado en una computadora, escala en función de la precisión de la computadora $\xi$ y la dimensión de correlación del atractor caótico, como $T \sim \xi ^ {- D / 2}$.
En \cite{shujun2005} se informaron algunos hallazgos sobre una nueva serie de indicadores dinámicos, que pueden reflejar cuantitativamente los efectos de la degradación en un mapa caótico digital realizado con una precisión finita de punto fijo, pero están restringidos a mapas caóticos lineales por partes $1D$ (PWLCM).
En \cite{Dias2011} se investigó el efecto de la precisión numérica sobre la distancia media y sobre el tiempo medio de coalescencia entre las trayectorias de los mapas determinísticos con un parámetro de ruido multiplicativo o con un término de ruido aditivo.
Nepomuceno \textit{et al.} \cite{Nepomuceno2017} estudió los cambios en las pseudo-órbitas de sistemas caóticos continuos al variar el tiempo y los esquemas de discretización.

Liao \cite{Liao2013b, Liao2012, Liao2009} propuso un algoritmo numérico llamado \textit{clean numerical simulation} (CNS).
Afirmó este enfoque numérico es extremadamente preciso para sistemas dinámicos caóticos en un intervalo finito dado.
Si la condición inicial fuera exacta, la predicción del caos a largo plazo sería posible en teoría.
Desafortunadamente, el tiempo de CPU requerido aumenta exponencialmente a medida que aumenta el número de dígitos de precisión y el orden de expansión de Taylor (para sistemas continuos), de modo que es prácticamente imposible dar verdaderas trayectorias de caos en un intervalo muy largo.
A diferencia de este análisis, que se lleva a cabo en un mapa, no abordamos el problema que surge de la discretización de tiempo.
Lo que es más, en \cite{Liao2009}, las precisiones más bajas son despreciadas porque están muy lejos de la trayectoria real.
En contraposición, nuestro enfoque proviene del punto de vista de la implementación en hardware, donde el uso de recursos mínimos es obligatorio.
Nuestro objetivo es investigar las características de cada precisión para que el diseñador tenga una visión general completa de las opciones que se utilizarán en su implementación.
De esta manera, que los diseñadores pueden decidir qué propiedades rescindir según los recursos y requisitos disponibles.

Una cosa importante a tener en cuenta es que además de analizar los cambios en las duraciones de los períodos, las propiedades estadísticas de las secuencias serán diferentes de las del sistema real, por lo que también deberían analizarse.
En \cite{Persohn2012} se desarrolló un excelente trabajo sobre las consecuencias que la precisión finita tiene sobre la periodicidad de un PRNG basado en el mapa logístico.
Allí, se determinaron el número, retardo y período de las órbitas del mapa logístico con diversos grados de precisión, sin embargo, carecían de un análisis estadístico.
Nuestra investigación complementa su trabajo al agregar cuantificadores estadísticos.
Lo que es más, analizaron la arquitectura de coma flotante del mapa, mientras que aquí hemos elegido la arquitectura de punto fijo ya que es la arquitectura óptima para las implementaciones en hardware.
Desde el punto de vista de la ingeniería, la aritmética de punto fijo es más eficiente que la de punto flotante, consume menos recursos y sus operaciones requieren un menor número de ciclos de reloj.
Como consecuencia, el consumo de energía también se ve disminuido.

Entre muchos sistemas caóticos disponibles en la literatura, estamos interesados en una familia de mapas $2D$ propuestos por Sprott \cite{Sprott1993}.
La principal característica de este sistema es que presenta múltiples atractores caóticos dependiendo del punto seleccionado en el espacio de los parámetros, esta característica es muy atractiva para ser utilizada en aplicaciones electrónicas.
Solo los resultados para la representación analítica de los mapas en \cite{Sprott1993} han sido publicados en la literatura abierta.

El objetivo de este documento es ampliar el análisis a la versión digital para posibilitar la implementación del hardware en aritmética de punto fijo.
Para lo cual es imprescindible conocer las dos características, la duración del período y el grado de aleatoriedad de las secuencias.
Desarrollamos un análisis detallado de la \textsl{degradación} del sistema caótico multiatractor a medida que se utiliza una implementación de punto fijo.
Por \textsl{degradación} queremos decir:
(a) la aparición de puntos fijos estables y órbitas periódicas estables con períodos cortos, dentro de un dominio de atracción de coma flotante sin órbitas estables;
(b) el atractor mismo se vuelve periódico y sus características estadísticas cambian, lo que hace que el sistema sea más determinista.

Las principales contribuciones de esta sección son:
\begin{itemize}
\item el análisis de los dominios de atracción de los atractores caóticos para un conjunto dado de parámetros a medida que aumenta el número de bits; en términos de la duración del período y la aparición de puntos fijos estables y órbitas periódicas con períodos cortos se consideran especialmente;
\item la determinación del consecuente umbral para el ancho del bus, para hacer que las propiedades estadísticas de la implementación digital sean cercanas a las de la implementación de coma flotante;
\item Dos funciones de distribución de probabilidad diferentes (PDF) se asignan para evaluar la estocasticidad de la serie temporal para diferentes anchos de bus.
Cada PDF $P$ se mide por la correspondiente entropía de Shannon normalizada $H(P)$.
Estas entropías tienen cambios abruptos en anchos de bus específicos.
Las duraciones de los períodos y el \textsl{MLE} también se evalúan y los resultados se comparan con las $H$s.
\end{itemize}