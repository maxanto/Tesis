\subsection{Introduction} \label{sec:intro}

Chaotic systems have an increasing number of applications and their implementation is specially involved due to the extreme sensitivity to initial conditions.
In general, these systems are used for the generation of controlled noises, these digital pseudo-random noise generators (PRNGs) can be employed in a large number of electronic applications, such as encryption sequences for privacy, multiplexing techniques, electromagnetic compatibility \cite{Machado2004,Smaoui2009,DeMicco2007A,DeMicco2007C,DeMicco2007B}.
In computers and digital devices only \textsl{pseudo chaotic} attractors can be generated.
But discretization may destroy the \textsl{pseudo chaotic} behavior and consequently is a nontrivial process.

Several strategies have been proposed in the literature for a correct selection of the optimal number of bits in hardware implementations.
However, most of these procedures are limited to linear systems \cite{Constantinides2002,Constantinides2003}.
In digital chaotic systems, a completely different behavior may be obtained by varying the precision.
This issue has gained interest recently, and several new schemes have been proposed \cite{Ding2007,Asseri2002,Azzaz2009}.

In short, in spite of the arithmetic used, i. e. fixed-point or floating-point arithmetic, the set of numbers that can be represented is limited. 
Even using extremely high precision as done by Liao and Wang \cite{Liao2013a} the generated sequences by a chaotic system using digital hardware will always be periodic.

Grebogi's work \cite{Grebogi1988} showed that the average length $T$  of periodic orbits of a dynamical system implemented in a computer, scales as a function of the computer precision $\xi$ and the correlation dimension of the chaotic attractor, as $T \sim  \xi^{-D/2}$.
In \cite{SHUJUN2005} some findings on a new series of dynamical indicators, which can quantitatively reflect the degradation effects on a digital chaotic map realized with a fixed-point finite precision, have been reported, but they are restricted to $1D$ piecewise linear chaotic maps (PWLCM).
In \cite{Dias2011} the effect of numerical precision on the mean distance and on the mean coalescence time between trajectories of deterministic maps with either multiplicative noise parameter or with an additive noise term was investigated.
Nepomuceno et al. \cite{Nepomuceno2017} studied the changes in the pseudo orbits of continuous chaotic systems when varying the time and discretization schemes.

Liao \cite{Liao2013b,Liao2012,Liao2009} proposed a numerical algorithm, namely the clean numerical simulation (CNS).
He claimed that the CNS gives an extremely precise numerical approach for chaotic dynamic systems in a given finite interval.
If the initial condition were exact, then the long-term prediction of chaos would be possible in theory, unfortunately, the required CPU time increases exponentially as the number of digits precision and Taylor expansion order increases (for continuous systems), so that it is practically impossible to give true trajectories of chaos in a very long interval.
Unlike our analysis, which is carried out on a map, we do not address the issue arising from the time discretization.
What is more, in \cite{Liao2009}, lower precisions are despised because they are very far from the real trajectory.
In contradistinction our approach comes from the hardware implementation point of view, where using minimum resources is mandatory, does not dismiss any precision.
Instead, our goal is to investigate the characteristics for each precision so that the designer has a complete overview of the options to be used in its implementation.
So that designers will be able to decide which properties to rescind according to the available resources and requirements.

One important thing to note is that besides analyzing the changes in the period lengths, the statistical properties of the sequences will be different from those of the real system and so they also should be analysed.
In \cite{Persohn2012} an excellent work about the consequences finite precision has on the periodicity of a PRNG based on the logistic map was developed.
There, the number, delay, and period of the orbits of the logistic map at varying degrees of precision were determined, however they lacked a statistical analysis.
Our research complements their work by adding statistical quantifiers.
What is more, they analysed the floating-point architecture of the map, while here we have chosen fixed-point architecture as it is the optimal architecture for hardware implementations.
From an engineering point of view, fixed-point arithmetic is more efficient than floating-point, it consumes fewer resources and their operations require lower number of clock cycles.
As a consequence, power consumption is also diminished.

Among many chaotic systems available in the literature, we are interested in a family of $2D$-maps proposed by Sprott \cite{Sprott1993}. 
The main characteristic of this system is it presents multiple chaotic attractors depending on the selected point in the parameter's space, this feature is very attractive to be used in electronic applications.
Only results for the analytical representation of the maps in \cite{Sprott1993} have been published in the open literature.

The objective of this paper is to extend the analysis to the digital version, to make possible  the hardware implementation in fixed-point arithmetic.
For which it is imperative to know both characteristics, period length and degree of randomness, of the sequences.
We developed a detailed analysis of the \textsl{degradation} of the multiattractor chaotic system as a fixed-point implementation is used.
By \textsl{degradation} we mean:
(a) the appearance of stable fixed points and stable periodic orbits with short periods, inside a floating-point domain of attraction without stable orbits;
(b) the attractor itself becomes periodic and its statistical characteristics change, making the system more deterministic.

The main contributions of this paper are: 
\begin{itemize}
\item the analysis of the domains of attraction of the chaotic attractors for a given set of parameters as the number of bits increases; in terms of period lengths and the appearance of stable fixed points and periodic orbits with short periods are specially considered;
\item the determination of the consequent \textsl{threshold width} for the bus, in order to make the statistical properties of the digital implementation close to those of the floating-point implementation; 
\item  two different probability distribution functions (PDF) are assigned  to evaluate the stochasticity of the time series for different bus widths.
Each PDF  $P$ is measured by the respective normalized Shannon entropy $H(P)$.
These entropies have abrupt changes at specific bus widths.
Period's lengths and \textsl{MLE} are also evaluated and results are compared with $H$s.
\end{itemize}