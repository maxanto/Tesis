\section{Conclusiones}

En este Capítulo se resumen los resultados de cuatro trabajos orientados a la implementación de sistemas caóticos.
Se propusieron y analizaron distintos sistemas y aplicaciones desde el punto de vista estadístico, siempre contemplando la implementación orientada a la ingeniería.

Con respecto a la exploración de caos en RNAs, encontramos que es posible conseguir caos robusto en una zona del espacio de los parámetros utilizando un sistema de tres neuronas.
Se obtuvo un MLE positivo en toda una zona del parámetro $P$, además la trayectoria del atractor no fue divergente, lo que confirma la existencia de caos para todo el rango.

Se desarrolló un nuevo método para el diseño de sistemas de criptocodificación mediante el empleo de mapas caóticos cuadráticos bidimensionales.
Se obtuvieron resultados muy prometedores hasta el momento mediante las simulaciones realizadas.
También se obtuvieron buenos resultados en la implementación en VHDL del sistema codificador, en la que se verificó que la salida generada por el sistema no cayera en ciclos periódicos debido a la utilización de presición finita.

A partir de los resultados presentados para PRNGs determinístico-estocásticos, es posible concluir que los los mejores PRNGs en cuanto a recursos y frecuencia de operación, resultaron ser los representados por aritmética entera.
La técnica de aleatorización de \textit{concatenado} hace que la calidad sea independiente de $\Delta t$ para esta representación numérica.
Para la técnica de aleatorización de \textit{descarte} se obtienen buenos resultados sólo para valores grandes de $\Delta t$.
Lo mismo ocurre con las implementaciones de punto fijo con ambas técnicas de aleatorización.
En términos de uso de recursos y limitaciones de frecuencia, el rendimiento en aritmética de enteros es considerablemente mejor que en punto flotante y punto fijo.
Se observó que para minimizar los recursos, se requiere un preprocesamiento de las ecuaciones del sistema caótico (escalado y polarización) para obtener divisores que sean potencia de $2$, tal como se explica en la Subsección \ref{sec:impleInt}.
Por otro lado, para el caso de la aritmética de punto flotante, el exponente que indica la posición de la coma se descarta en todas las técnicas de aleatorización utilizadas y, en consecuencia, la dinámica del sistema se ve muy perturbada por el proceso de aleatorización.
Entonces, como se muestra en el Cuadro \ref{tabla:TablaImpLorenz2}, $\Delta t $ no es la variable relevante para predecir si un PRNG será bueno o malo.

Finalmente, se desarrolló un análisis detallado de los cambios en el comportamiento de una implementación en punto fijo de un mapa cuadrático bidimensional.
El objetivo fue reportar la tasa de degradación de las propiedades de cada sistema, para que para que puedan ser utilizadas en el diseño de aplicaciones particulares.
Los resultados muestran que es posible determinar un umbral para el número de bits empleados en la representación de punto fijo del sistema, mientras que el dominio de atracción conserva su integridad y se mantienen las características de las secuencias generadas.
Con la ayuda de los cuantificadores de aleatoriedad introducidos fue posible determinar ese límite, en el caso del estudio fue de 23 bits.
El mismo procedimiento debe repetirse para cualquier otro sistema si se desea utilizarlo en una aplicación electrónica digital, como generadores de ruido controlados o para desarrollar nuevos sistemas de encriptación.