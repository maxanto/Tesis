\section{Sistemas Caóticos}

Ha quedado claro que existen sistemas deterministas que rompen con el preconcepto de que los sistemas físicos pueden clasificarse en dos conjuntos disjuntos: sistemas deterministas y sistemas estocásticos.
En esa concepción antigua un sistema determinista es aquél para el cual conocemos el modelo y por lo tanto es posible predecir con exactitud la evolución de sus variables de estado.
Se utilizan en su descripción ecuaciones diferenciales o de recurrencia.
Por otra parte un sistema estocástico es aquél para el cual el modelo no se conoce o se lo supone sumamente complejo como para ser obtenido, de modo que se adopta la estrategia de estudiar sus variables de estado en forma estadística.
Se utilizan entonces en la descripción ecuaciones diferenciales o de recurrencia estocásticas.

El caos determinista demostró que complejidad en la evolución temporal no es sinónimo de complejidad en el modelo, cuando hay no linealidad: modelos deterministas muy simples
originan señales de aspecto estocástico.
La sensibilidad a las condiciones iniciales hace que en estos sistemas la predictibilidad sea a corto plazo (luego de un tiempo finito es imposible predecir la evolución) lo que ubica a estos sistemas en una posición intermedia entre determinista y estocástico.

Como consecuencia se desarrollaron en los últimos años un número creciente de aplicaciones de los sistemas caóticos, empleándolos principalmente como generadores de
ruido controlado, generadores de números pseudoaleatorios, portadoras de señales, sistemas de encriptado, etc.

Hoy en día, los sistemas dinámicos son un objeto de estudio interdisciplinario, aunque originalmente fue una rama de la física.
Todo comenzó a mediados del 1600, cuando Newton inventó las ecuaciones diferenciales, descubriendo sus leyes del movimiento de gravitación universal, y las combinó con las leyes de Kepler sobre el movimiento planetario.
Específicamente, Newton resolvió el problema de los dos cuerpos (por ejemplo el sistema tierra-sol).

Subsecuentes generaciones de matemáticos y físicos intentaron extender los métodos analíticos de Newton al problema de los tres cuerpos (por ejemplo luna-tierra-sol), pero curiosamente para resolver este problema se necesitó mucho más esfuerzo.
Luego de décadas de esfuerzo, se dieron cuenta de que el problema de los tres cuerpos era esencialmente imposible de resolver, en el sentido de obtener las fórmulas explícitas.

La ruptura vino con el trabajo de Poincaré a finales del 1800.
Él introdujo un nuevo punto de vista que enfatizaba las cuestiones cualitativas más que las cuantitativas (por ejemplo, ¿es estable el sistema luna-tierra-sol?).
Poincaré desarrolló una poderosa aproximación geométrica que es usada hoy para estudiar sistemas dinámicos y también fue el primero en vislumbrar la posibilidad
de caos, en el cual un sistema determinístico exhibe un comportamiento aperiódico que depende sensiblemente de las condiciones iniciales, haciendo así imposible la predicción a largo plazo.

Pero el caos se mantuvo en segundo plano hasta la segunda mitad del 1900, en donde los osciladores no lineales jugaron un rol vital en el desarrollo de tecnologías de radio, radar, lazos de enganche de fase y láser.
Por el lado matemático, los osciladores no lineales también estimularon la invención de nuevas técnicas matemáticas.
Los métodos geométricos de Poincaré se fueron extendiendo para producir un conocimiento mucho más profundo de la mecánica clásica.

La invención de la computadora por el 1950 fue una línea divisoria en la historia de los sistemas dinámicos.
La computadora nos permite experimentar con ecuaciones en una forma que antes era imposible, y así desarrollar alguna intuición acerca de los sistemas no lineales.
Estos experimentos llevaron a Lorenz a descubrir en 1963 el movimiento caótico de un atractor extraño, mientras estudiaba un modelo simplificado de la circulación de convexión para comprender mejor la notoria impredictibilidad del clima.
Lorenz encontró que la solución a sus ecuaciones nunca caían al equilibrio o a un estado periódico.
Además, si comenzaba sus simulaciones de dos condiciones iniciales ligeramente diferentes, los comportamientos resultantes pronto serían
totalmente diferentes.
Como consecuencia de ello, el sistema es inherentemente impredecible, pequeños errores en las mediciones del estado actual de la atmósfera (o cualquier sistema caótico)
sería amplificado rápidamente.
Pero Lorenz también mostró que había estructura en el caos, cuando fueron ploteadas en tres dimensiones, las soluciones a sus ecuaciones cayeron sobre un
set de puntos en forma de mariposa.
Él sostuvo que este sistema tenía que ser “un infinito complejo de superficies”.
Lo que hoy podríamos considerar como un ejemplo de fractal.

El trabajo de Lorenz tuvo un pequeño impacto hasta 1970, los años del boom del caos.
Se desarrollaron teorías completamente nuevas basadas en consideraciones sobre atractores caóticos, como turbulencia de fluidos y biología de las poblaciones y se encontraron comportamientos caóticos en reacciones químicas, circuitos electrónicos, osciladores mecánicos, semiconductores y oscilaciones biológicas como el ritmo cardíaco y circadiano.

Hoy, la teoría del caos es un herramienta más para el estudio de sistemas dinámicos y los sistemas caóticos forman parte de una gran cantidad de dispositivos.

%ESTO ESTÁ SACADO DE IC, LO TENGO QUE ACOMODAR
En algunas aplicaciones puede interesar, más que conocer las ecuaciones explícitas de las soluciones de un sistema, poder analizar sus propiedades cualitativas, tales como la periodicidad, el comportamiento cuando crece la variable independiente (la que se supone que es el tiempo), si es constante, o si se aproxima a una solución conocida, etc. Una herramienta útil en este sentido es el diagrama de fase. El espacio de fase es el lugar geométrico que ocupan las posibles soluciones del sistema de ecuaciones diferenciales, en él se dibujan las trayectorias que son solución a un sistema de ecuaciones. La teoría cualitativa intenta clasificar los sistemas en función del tipo de trayectorias que poseen, en lugar de intentar resolver las ED’s.

Sin embargo, el teorema de existencia y unicidad de las soluciones a un sistema de ecuaciones garantiza que si f es continuamente diferenciable, los campos vectoriales sobre el espacio de fases son suaves y cada punto de este espacio tiene solución única. La existencia de este teorema tiene un corolario importante: trayectorias diferentes nunca se intersectan. Como consecuenciade esto, las trayectorias sobre el plano de fases quedan restringidas a: un nodo estable o inestable, centro, foco estable o inestable y puerto.
Queda claro, entonces que un sistema continuo con derivada continua debe tener dimensión mayor o igual a tres para que pueda ser caótico, además, la trayectoria en el espacio de fases debe ocupar un dominio restringido.

